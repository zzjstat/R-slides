\documentclass[10pt,compress,t,noamsthm,notheorem,table,handout]{ctexbeamer}
\usetheme{Boadilla}
\useinnertheme{circles}
\useoutertheme{shadow}
\usecolortheme{seahorse}
\usefonttheme[onlymath]{serif}
\setbeamertemplate{navigation symbols}{}
% \setbeamercovered{transparent}
\usepackage{natbib}
\usepackage{amsthm}
\usepackage{makecell}
\usepackage{multirow}
\renewcommand{\qedsymbol}{$\blacksquare$}
\bibliographystyle{unsrtnat}
\usepackage{color}
\setbeamercolor{myfootline}{bg=white,fg=blue}
\definecolor{myfoot}{rgb}{0.5,0.2,0.5}
\definecolor{darkblue}{rgb}{0.1,0,0.85}
\setbeamertemplate{headline}
  { \leavevmode\begin{beamercolorbox}[wd=\paperwidth,ht=1.25ex,dp=1ex,left]{}
    \end{beamercolorbox}}
\setbeamertemplate{footline}% 自定义页脚
  { \leavevmode\mbox{%
    \begin{beamercolorbox}[wd=.75\paperwidth,ht=2.25ex,dp=1ex,left]{myfootline}%
        \rule{2em}{0pt}\color{myfoot}\ttfamily\scriptsize%
        %\insertshortauthor~(\insertshortinstitute)
    \end{beamercolorbox}%
    \begin{beamercolorbox}[wd=.25\paperwidth,ht=2.25ex,dp=1ex,right]{myfootline}%
       {\color{myfoot}\ttfamily\scriptsize\insertframenumber{}/%
        \inserttotalframenumber\hspace*{3ex}}
    \end{beamercolorbox}}
    \vskip0pt }

\setbeamercolor{frametitle}{fg=blue,bg=white}
\setbeamertemplate{frametitle}{%
  \leavevmode\linespread{1}\large\textbf{\insertframetitle}\par
  \color{structure.fg!30!bg}\rule[6pt]{\linewidth}{2pt}\par\vspace{-1.0em}}

% \setbeamertemplate{blocks}[default] % beamer块(含定理类环境)不要阴影
\setbeamertemplate{bibliography entry title}{}{}
\setbeamertemplate{bibliography entry location}{}{}
\setbeamertemplate*{bibliography entry note}{}{}
\setbeamersize{text margin left=0.75cm, text margin right=0.75cm}

\setbeamercolor{bluebox}{fg=black,bg=blue!10}
\setbeamercolor{redbox}{fg=black,bg=red!10}
\newenvironment{Boxblue}[1][\textwidth]
  {\begin{beamercolorbox}[sep=0.1em,shadow=true,wd=#1,rounded=true,center]{bluebox}}
  {\end{beamercolorbox}}
\newenvironment{Boxred}[1][\textwidth]
  {\begin{beamercolorbox}[sep=0.1em,shadow=true,wd=#1,rounded=true,center]{redbox}}
  {\end{beamercolorbox}}
\usepackage{stmaryrd}
\usepackage{amsmath,amssymb,amsfonts,bm}
\usepackage{graphicx,xcolor}
\graphicspath{{figures/}}
\usepackage{hyperref}
\hypersetup{pdfborder=001,colorlinks=true,linkcolor=darkblue,urlcolor=blue}
\usepackage{bbding}
\newcommand{\Bullet}{{\fontsize{6pt}{6pt}\selectfont\CircleSolid}}
\newcommand{\Hand}{{\fontsize{8pt}{6pt}\selectfont\HandRight}}
\newcommand{\zhu}{{\color{blue!40}\Bullet}}
\newcommand{\zhuu}{{\color{red!80}\Hand}}
\newcommand{\labeli}{\zhu}
\newenvironment{blist}%
    {\begin{list}{{\hfill\raisebox{1.12pt}{\color{blue!60}\zhu}}}{%
     \leftmargin2em\labelwidth1.5em\labelsep0.5em
     \itemsep1ex\itemindent0pt\parsep0pt\topsep0pt}}
    {\end{list}}
\newenvironment{myitem}
  {\begin{list}{{\hfill\raisebox{0pt}{\labeli}}}{%
    \setlength{\leftmargin}{1.2em}\labelwidth0.8em\labelsep.4em%
    \itemsep1ex\parsep2pt\itemindent0pt\topsep0pt}}{\end{list}}
\newenvironment{subitem}
  {\begin{list}{{\hfill\raisebox{0pt}{-}}}{%
    \setlength{\leftmargin}{1.2em}\labelwidth0.8em\labelsep.4em%
    \itemsep0ex\parsep2pt\itemindent0pt\topsep0pt}}{\end{list}}
\usepackage{colortbl}
\usepackage{tikz}
\usetikzlibrary{arrows}
\usepackage{stmaryrd}
\usepackage{amsfonts}
\usepackage[ruled,linesnumbered]{algorithm2e}
\usepackage{float} 
\usepackage{booktabs}
\usepackage[framemethod=tikz]{mdframed}
\newmdenv[linecolor=green,middlelinewidth=1pt,%
          roundcorner=3pt,backgroundcolor=white,%
          innertopmargin=0.8em,innerbottommargin=0.5em,%
          innerleftmargin=3pt,innerrightmargin=3pt,%
          skipbelow=0.5em,skipabove=1em,%
          splittopskip=\topskip]{Block}
\newmdenv[linecolor=green,middlelinewidth=1pt,%
          roundcorner=3pt,backgroundcolor=red!5!white,%
          innertopmargin=0.5em,innerbottommargin=0.5em,%
          innerleftmargin=3pt,innerrightmargin=3pt,%
          skipbelow=0.5em,skipabove=1em,%
          splittopskip=\topskip]{redbox}
\newmdenv[linecolor=green,middlelinewidth=0.5pt,%
          %outerlinewidth=0.5pt,skipabove=0pt,
          roundcorner=3pt,backgroundcolor=white,%
          innerbottommargin=3pt,innerrightmargin=5pt,%
          innerleftmargin=5pt,leftmargin=0ex]{mathbox}
\newmdenv[linecolor=blue!5!green,middlelinewidth=0.5pt,%
          roundcorner=3pt,backgroundcolor=yellow!5,%
          % frametitle={Hello},frametitlebackgroundcolor=green!50,%
          % skipabove=2pt,skipbelow=2pt,%
          innerleftmargin=3pt,leftmargin=0ex]{notebox}
\newmdenv[linecolor=white,font={\scriptsize},%
          fontcolor=blue!85,backgroundcolor=yellow!5,%
          skipabove=1ex,skipbelow=0pt,innerbottommargin=0.5ex,%
          innerleftmargin=3pt,leftmargin=1em]{myref}

%%%%%%%%%%%%%%%%%%%%%%%%%%%%%%%%%%%%%%%%%%%%%%%%%%%%%%%%%%%%%%%%%%%%%%%%%%%%%%
\renewcommand{\thefootnote}{}% 不要编号
\setbeamertemplate{footnote}{% 首行不缩进
  \noindent\insertfootnotemark%
  \scriptsize\color{blue!85!green!85}\insertfootnotetext\par\kern1ex}
\renewcommand\footnoterule%    更改横线属性:长度,粗细,颜色
  {\color{red}\kern-3pt\rule{0.4\linewidth}{0.5pt}\par\kern2.6pt}
%%%%%%%%%%%%%%%%%%%%%%%%%%%%%%%%%%%%%%%%%%%%%%%%%%%%%%%%%%%%%%%%%%%%%%%%%%%%%%

\usepackage[many]{tcolorbox}
\tcbset{highlight math %
  style={enhanced, colframe=blue!40,colback=yellow!20,arc=4pt,boxrule=1pt}}
\newtcbox{\subsubtit}[1][]{%
  after skip=1em,boxrule=0.5pt,
  fontupper=\color{blue}\bfseries,top=0.5ex,bottom=0.5ex,
  left=1ex,right=1ex,
  colframe=green,colback=red!5!white,#1}

%\renewcommand{\baselinestretch}{1.1}
\linespread{1.1}
\setlength{\parskip}{1ex}

\usepackage{listings} %插入代码
\usepackage{xcolor}
\lstset{numbers=left, %设置行号位置
        numberstyle=\tiny, %设置行号大小
        keywordstyle=\color{blue}, %设置关键字颜色
        commentstyle=\color[cmyk]{1,0,1,0}, %设置注释颜色
        frame=single, %设置边框格式
        escapeinside=``, %逃逸字符(1左面的键),用于显示中文
        %breaklines, %自动折行
        extendedchars=false, %解决代码跨页时,章节标题,页眉等汉字不显示的问题
        xleftmargin=2em,xrightmargin=2em, aboveskip=1em, %设置边距
        tabsize=4, %设置tab空格数
        showspaces=false, %不显示空格
        basicstyle=\ttfamily,
       }
\begin{document}

%%%%% =======================================================================
\title{R语言-4}
\author{郑泽靖 \and zzjstat2023@163.com }
\institute{\normalsize 北京师范大学统计学院}
\date{\today}

% ===== title page =====
\begin{frame}[plain]
  \titlepage
\end{frame}

% ===== contents =====
% \begin{frame}
%  \frametitle{Outline}
%  \tableofcontents[hideallsubsections] %[pausesections]
% \end{frame}
\begin{frame}[fragile]{R语言中的条件语句}
    在 R 语言中,常用的条件控制语句是 \texttt{if} 语句,它有三种主要形式。
    
    \begin{itemize}
        \item \textbf{(1) 单条件 (if)}:基础判断。
        \item \textbf{(2) 复合条件 (if/else)}:二选一判断。
        \item \textbf{(3) 多条件 (if/else if/else)}:多重分支判断。
    \end{itemize}
\end{frame}

\begin{frame}[fragile]{单条件语句 (if)}
    \textbf{调用格式:}
    \begin{lstlisting}[language=R]
if (条件表达式) {
    # 仅当条件表达式为 TRUE 时执行
    程序代码
}
    \end{lstlisting}
    
    \textbf{执行过程}:
    若“条件表达式”成立(结果为 \texttt{TRUE}),则执行“程序代码”;否则,跳过执行。
\end{frame}

\begin{frame}[fragile]{单条件语句示例}
    \begin{lstlisting}[language=R]
x <- 1  # 初始值

if (x < 1) {  # 1 < 1 为 FALSE
    x <- x + 1  
}
print(x)  # x 仍为 1

if (x < 2) {  # 1 < 2 为 TRUE
    x <- x + 1  
}
print(x)  # x 变为 2
    \end{lstlisting}
    
    \textbf{要点提示}:条件表达式必须返回一个**长度为 1 的逻辑值**(\texttt{TRUE} 或 \texttt{FALSE})。若输入长度大于 1 的向量,将仅检查第一个元素,并给出警告。
\end{frame}

\begin{frame}[fragile]{复合条件语句 (if/else)}
    \textbf{调用格式:}
    \begin{lstlisting}[language=R]
if (条件表达式) {
    程序代码1  # 条件成立时执行
} else {
    程序代码2  # 条件不成立时执行
}
    \end{lstlisting}
    
    \textbf{注意事项}:
    \begin{itemize}
        \item \texttt{else} 必须紧跟在 \texttt{if} 代码块的右花括号 \texttt{\}} 之后,并在同一行。
    \end{itemize}
\end{frame}

\begin{frame}[fragile]{多条件语句 (if/else if/else)}
    \textbf{调用格式:}
    \begin{lstlisting}[language=R]
if (条件表达式1) {
    程序代码1
} else if (条件表达式2) {
    程序代码2
} else {
    程序代码3  # 以上条件都不满足时执行
}
    \end{lstlisting}
    
    \textbf{执行过程}:
    程序依次检查条件表达式,一旦发现某个条件成立,则执行相应的程序代码,**然后跳出整个结构**。
\end{frame}

\begin{frame}[fragile]{多条件语句示例:判断数字符号}
    编写代码判断一个数字是正数、负数还是零。
    \begin{lstlisting}[language=R]
x <- -5
sign_status <- ""

if (x > 0) {
    sign_status <- "正数"
} else if (x < 0) { # 检查是否为负数
    sign_status <- "负数" 
} else {            # 既不大于 0 也不小于 0,则为 0
    sign_status <- "零"
}
print(sign_status) # 输出 "负数"
    \end{lstlisting}
\end{frame}


\begin{frame}{循环语句概述}
    循环语句用于重复执行一段程序代码,是实现迭代和批量操作的重要手段。R 语言主要提供两种循环语句:
    \begin{itemize}
        \item \textbf{for 循环}:适用于已知重复次数或需要遍历向量元素的场景。
        \item \textbf{while 循环}:适用于重复次数不确定,依赖条件表达式来控制终止的场景。
    \end{itemize}
\end{frame}

\begin{frame}[fragile]{for 循环语句}
    \textbf{调用格式:}
    \begin{lstlisting}[language=R]
for(i in 向量v) {
    # 循环体 expr
}
    \end{lstlisting}
    
    \textbf{执行过程}:
    \begin{enumerate}
        \item 循环变量 $i$ 依次取向量 $v$ 中的每一个元素值。
        \item 每取一个值,就执行一次循环体内的程序代码。
    \end{enumerate}
\end{frame}

\begin{frame}[fragile]{while 循环语句}
    \textbf{调用格式:}
    \begin{lstlisting}[language=R]
while(条件表达式) {
    循环体 
    # 确保循环体内有能改变条件表达式的代码,防止死循环
}
    \end{lstlisting}
    
    \textbf{执行过程}:
    \begin{enumerate}
        \item 计算条件表达式。
        \item 若为 \texttt{TRUE},则运行循环体,然后返回第 1 步。
        \item 若为 \texttt{FALSE},则结束循环。
    \end{enumerate}
\end{frame}

\begin{frame}[fragile]{示例:模拟随机变量 $X$ (掷骰子直到 6 点)}
    模拟 $X$: 掷骰子,直到掷出 6 点为止,记录掷骰子的次数 $X$。
    
    \textbf{使用 \texttt{while} 循环和 \texttt{break} 语句:}
    \begin{lstlisting}[language=R]
X_observations <- numeric(10) # 存储 10 次观测值
for(i in 1:10) { 
    n <- 0  # 实验次数
    while(TRUE) { # 设置无限循环
        n <- n + 1 
        y <- sample(1:6, 1)  # 模拟掷骰子
        if (y == 6) {        # 满足终止条件
            X_observations[i] <- n
            break            # 终止 while 循环
        }
    }
}
print(X_observations) # 显示 10 次模拟结果
    \end{lstlisting}
\end{frame}

\begin{frame}[fragile]{函数 (Function) 定义}
    函数是封装代码逻辑,实现代码重用和模块化的重要工具。
    \begin{lstlisting}[language=R]
name <- function(arg1, arg2, ...) {
    # 函数体:程序代码
    return(result) # 可选,但推荐使用
}
    \end{lstlisting}
    
    \begin{itemize}
        \item \textbf{name}:自定义函数名。
        \item \textbf{arg1, ...}:函数的输入参数。
        \item \textbf{return(result)}:显式指定函数返回值。若省略 \texttt{return},默认返回函数体中最后一行代码的计算结果。
    \end{itemize}
\end{frame}

\begin{frame}[fragile]{函数示例:计算前 $n$ 个自然数之和}
    使用公式法定义函数 \texttt{mysum}:
    \begin{lstlisting}[language=R]
mysum <- function(n) {  
    # 检查输入是否为正整数
    if (n < 1 | n != round(n)) {
        stop("参数n必须是正整数。")
    }
    s <- n * (n + 1) / 2  
    return(s)  
}

# 调用示例
print(mysum(50))  
# 输出 [1] 1275
    \end{lstlisting}
\end{frame}


\begin{frame}[fragile]{定积分计算:\texttt{integrate} 函数}
    R 语言通过内置的 \texttt{integrate} 函数进行一元函数的数值积分。
    
    计算定积分 $\int_0^1 x^2 \mathrm{~d} x$:
    \begin{lstlisting}[language=R]
# integrate 的第一个参数是一个函数对象
result <- integrate(function(x) x^2, 
                     lower = 0, upper = 1)
print(result)
    \end{lstlisting}
    运行结果:
    \[
    \text{\$value} = 0.3333333 \text{ with absolute error < } 3.7 \times 10^{-15}
    \]
    \small{注意:积分的返回值是一个包含结果和误差信息的列表。}
\end{frame}

\begin{frame}[fragile]{均匀分布 $U(a, b)$ 的期望值}
    设随机变量 $\xi \sim U(0, 60)$,理论期望值 $\mathrm{E}(\xi) = \frac{0+60}{2} = 30$。
    
    使用积分计算 $\mathrm{E}(\xi) = \int_{-\infty}^{\infty} x \cdot f(x) \mathrm{d}x$:
    \begin{lstlisting}[language=R]
# f(x) 为均匀分布的密度函数 dunif(x, min, max)
result <- integrate(function(x) 
              x * dunif(x, min = 0, max = 60), 
              lower = -Inf, upper = Inf)
print(result$value)
    \end{lstlisting}
    
    \textbf{计算结果:} 
    \[
    \mathrm{E}(\xi) \approx 30.00000 \quad \text{(高精度逼近)}
    \]
\end{frame}

\begin{frame}{概率分布函数概述 (d/p/q/r)}
    在 R 中,大多数概率分布函数都遵循 **d/p/q/r** 的命名规则:
    \begin{itemize}
        \item \texttt{d\textbf{xxx}(x)}: \textbf{D}ensity / \textbf{D}istribution function (概率密度或概率质量函数)。
        \item \texttt{p\textbf{xxx}(q)}: \textbf{P}robability function (累积分布函数 CDF),表示 $ P(X \leq q) $。
        \item \texttt{q\textbf{xxx}(p)}: \textbf{Q}uantile function (分位数函数),求解 $ F(x) = p $ 的 $ x $ 值。
        \item \texttt{r\textbf{xxx}(n)}: \textbf{R}andom number generation (随机数生成函数)。
    \end{itemize}
    \small{(xxx 为分布的缩写,如 `norm`, `binom`, `unif` 等)}
\end{frame}

\begin{frame}[fragile]{二项分布(Binomial)示例}
    \begin{itemize}
        \item \texttt{dbinom(x, size, prob)}
        \item \texttt{pbinom(q, size, prob)}
        \item \texttt{rbinom(n, size, prob)}
    \end{itemize}
    \begin{lstlisting}[language=R]
> dbinom(5, size = 10, prob = 0.5)  
# P(X=5) = 0.246
> pbinom(5, size = 10, prob = 0.5)  
# P(X<=5) = 0.623
> rbinom(2, size = 10, prob = 0.5)  
# [1] 7 8 (示例值)
    \end{lstlisting}
\end{frame}

\begin{frame}[fragile]{正态分布(Normal)示例}
    \begin{itemize}
        \item \texttt{dnorm(x, mean, sd)}
        \item \texttt{pnorm(q, mean, sd)}
        \item \texttt{rnorm(n, mean, sd)}
    \end{itemize}
    \begin{lstlisting}[language=R]
> dnorm(0.5, mean = 0, sd = 1)   
# 0.3520...
> pnorm(1.96, mean = 0, sd = 1)  
# P(Z<=1.96) = 0.975
> rnorm(3, mean = 0, sd = 1)     
# [1] -0.62... 1.03... 0.70...
    \end{lstlisting}
\end{frame}

\begin{frame}[fragile]{练习题:斐波那契数列函数}
    编写一个名为 \texttt{fibonacci} 的 R 函数,该函数接受一个正整数 $n$ 作为输入参数,并返回一个长度为 $n$ 的斐波契数列。
    
    \textbf{斐波那契数列定义:}
    \[ F_0=0, F_1=1, F_n = F_{n-1} + F_{n-2} \quad (n \ge 2) \]
    
    \textbf{示例:}
    如果输入 \texttt{n = 5},则输出应为:
    \begin{verbatim}
[1] 0 1 1 2 3
    \end{verbatim}
    
    \textbf{提示:}
    \begin{itemize}
        \item 使用 \texttt{numeric()} 函数初始化结果向量。
        \item 使用 \texttt{if} 语句处理 $n=1$ 和 $n=2$ 的边界情况。
        \item 使用 \texttt{for} 循环来计算 $n \ge 3$ 的值。
    \end{itemize}
\end{frame}

\begin{frame}{}
\centering \Huge
  \emph{Thanks!}
\end{frame}

\end{document}
