\documentclass[10pt,compress,t,noamsthm,notheorem,table,handout]{ctexbeamer}
\usetheme{Boadilla}
\useinnertheme{circles}
\useoutertheme{shadow}
\usecolortheme{seahorse}
\usefonttheme[onlymath]{serif}
\setbeamertemplate{navigation symbols}{}
% \setbeamercovered{transparent}
\usepackage{natbib}
\usepackage{amsthm}
\usepackage{makecell}
\usepackage{multirow}
\renewcommand{\qedsymbol}{$\blacksquare$}
\bibliographystyle{unsrtnat}
\usepackage{color}
\setbeamercolor{myfootline}{bg=white,fg=blue}
\definecolor{myfoot}{rgb}{0.5,0.2,0.5}
\definecolor{darkblue}{rgb}{0.1,0,0.85}
\setbeamertemplate{headline}
  { \leavevmode\begin{beamercolorbox}[wd=\paperwidth,ht=1.25ex,dp=1ex,left]{}
    \end{beamercolorbox}}
\setbeamertemplate{footline}% 自定义页脚
  { \leavevmode\mbox{%
    \begin{beamercolorbox}[wd=.75\paperwidth,ht=2.25ex,dp=1ex,left]{myfootline}%
        \rule{2em}{0pt}\color{myfoot}\ttfamily\scriptsize%
        %\insertshortauthor~(\insertshortinstitute)
    \end{beamercolorbox}%
    \begin{beamercolorbox}[wd=.25\paperwidth,ht=2.25ex,dp=1ex,right]{myfootline}%
       {\color{myfoot}\ttfamily\scriptsize\insertframenumber{}/%
        \inserttotalframenumber\hspace*{3ex}}
    \end{beamercolorbox}}
    \vskip0pt }

\setbeamercolor{frametitle}{fg=blue,bg=white}
\setbeamertemplate{frametitle}{%
  \leavevmode\linespread{1}\large\textbf{\insertframetitle}\par
  \color{structure.fg!30!bg}\rule[6pt]{\linewidth}{2pt}\par\vspace{-1.0em}}

% \setbeamertemplate{blocks}[default] % beamer块(含定理类环境)不要阴影
\setbeamertemplate{bibliography entry title}{}{}
\setbeamertemplate{bibliography entry location}{}{}
\setbeamertemplate*{bibliography entry note}{}{}
\setbeamersize{text margin left=0.75cm, text margin right=0.75cm}

\setbeamercolor{bluebox}{fg=black,bg=blue!10}
\setbeamercolor{redbox}{fg=black,bg=red!10}
\newenvironment{Boxblue}[1][\textwidth]
  {\begin{beamercolorbox}[sep=0.1em,shadow=true,wd=#1,rounded=true,center]{bluebox}}
  {\end{beamercolorbox}}
\newenvironment{Boxred}[1][\textwidth]
  {\begin{beamercolorbox}[sep=0.1em,shadow=true,wd=#1,rounded=true,center]{redbox}}
  {\end{beamercolorbox}}
\usepackage{stmaryrd}
\usepackage{amsmath,amssymb,amsfonts,bm}
\usepackage{graphicx,xcolor}
\graphicspath{{figures/}}
\usepackage{hyperref}
\hypersetup{pdfborder=001,colorlinks=true,linkcolor=darkblue,urlcolor=blue}
\usepackage{bbding}
\newcommand{\Bullet}{{\fontsize{6pt}{6pt}\selectfont\CircleSolid}}
\newcommand{\Hand}{{\fontsize{8pt}{6pt}\selectfont\HandRight}}
\newcommand{\zhu}{{\color{blue!40}\Bullet}}
\newcommand{\zhuu}{{\color{red!80}\Hand}}
\newcommand{\labeli}{\zhu}
\newenvironment{blist}%
    {\begin{list}{{\hfill\raisebox{1.12pt}{\color{blue!60}\zhu}}}{%
     \leftmargin2em\labelwidth1.5em\labelsep0.5em
     \itemsep1ex\itemindent0pt\parsep0pt\topsep0pt}}
    {\end{list}}
\newenvironment{myitem}
  {\begin{list}{{\hfill\raisebox{0pt}{\labeli}}}{%
    \setlength{\leftmargin}{1.2em}\labelwidth0.8em\labelsep.4em%
    \itemsep1ex\parsep2pt\itemindent0pt\topsep0pt}}{\end{list}}
\newenvironment{subitem}
  {\begin{list}{{\hfill\raisebox{0pt}{-}}}{%
    \setlength{\leftmargin}{1.2em}\labelwidth0.8em\labelsep.4em%
    \itemsep0ex\parsep2pt\itemindent0pt\topsep0pt}}{\end{list}}
\usepackage{colortbl}
\usepackage{tikz}
\usetikzlibrary{arrows}
\usepackage{stmaryrd}
\usepackage{amsfonts}
\usepackage[ruled,linesnumbered]{algorithm2e}
\usepackage{float} 
\usepackage{booktabs}
\usepackage[framemethod=tikz]{mdframed}
\newmdenv[linecolor=green,middlelinewidth=1pt,%
          roundcorner=3pt,backgroundcolor=white,%
          innertopmargin=0.8em,innerbottommargin=0.5em,%
          innerleftmargin=3pt,innerrightmargin=3pt,%
          skipbelow=0.5em,skipabove=1em,%
          splittopskip=\topskip]{Block}
\newmdenv[linecolor=green,middlelinewidth=1pt,%
          roundcorner=3pt,backgroundcolor=red!5!white,%
          innertopmargin=0.5em,innerbottommargin=0.5em,%
          innerleftmargin=3pt,innerrightmargin=3pt,%
          skipbelow=0.5em,skipabove=1em,%
          splittopskip=\topskip]{redbox}
\newmdenv[linecolor=green,middlelinewidth=0.5pt,%
          %outerlinewidth=0.5pt,skipabove=0pt,
          roundcorner=3pt,backgroundcolor=white,%
          innerbottommargin=3pt,innerrightmargin=5pt,%
          innerleftmargin=5pt,leftmargin=0ex]{mathbox}
\newmdenv[linecolor=blue!5!green,middlelinewidth=0.5pt,%
          roundcorner=3pt,backgroundcolor=yellow!5,%
          % frametitle={Hello},frametitlebackgroundcolor=green!50,%
          % skipabove=2pt,skipbelow=2pt,%
          innerleftmargin=3pt,leftmargin=0ex]{notebox}
\newmdenv[linecolor=white,font={\scriptsize},%
          fontcolor=blue!85,backgroundcolor=yellow!5,%
          skipabove=1ex,skipbelow=0pt,innerbottommargin=0.5ex,%
          innerleftmargin=3pt,leftmargin=1em]{myref}

%%%%%%%%%%%%%%%%%%%%%%%%%%%%%%%%%%%%%%%%%%%%%%%%%%%%%%%%%%%%%%%%%%%%%%%%%%%%%%
\renewcommand{\thefootnote}{}% 不要编号
\setbeamertemplate{footnote}{% 首行不缩进
  \noindent\insertfootnotemark%
  \scriptsize\color{blue!85!green!85}\insertfootnotetext\par\kern1ex}
\renewcommand\footnoterule%    更改横线属性:长度,粗细,颜色
  {\color{red}\kern-3pt\rule{0.4\linewidth}{0.5pt}\par\kern2.6pt}
%%%%%%%%%%%%%%%%%%%%%%%%%%%%%%%%%%%%%%%%%%%%%%%%%%%%%%%%%%%%%%%%%%%%%%%%%%%%%%

\usepackage[many]{tcolorbox}
\tcbset{highlight math %
  style={enhanced, colframe=blue!40,colback=yellow!20,arc=4pt,boxrule=1pt}}
\newtcbox{\subsubtit}[1][]{%
  after skip=1em,boxrule=0.5pt,
  fontupper=\color{blue}\bfseries,top=0.5ex,bottom=0.5ex,
  left=1ex,right=1ex,
  colframe=green,colback=red!5!white,#1}

%\renewcommand{\baselinestretch}{1.1}
\linespread{1.1}
\setlength{\parskip}{1ex}

\usepackage{listings} %插入代码
\usepackage{xcolor}
\lstset{numbers=left, %设置行号位置
        numberstyle=\tiny, %设置行号大小
        keywordstyle=\color{blue}, %设置关键字颜色
        commentstyle=\color[cmyk]{1,0,1,0}, %设置注释颜色
        frame=single, %设置边框格式
        escapeinside=``, %逃逸字符(1左面的键),用于显示中文
        %breaklines, %自动折行
        extendedchars=false, %解决代码跨页时,章节标题,页眉等汉字不显示的问题
        xleftmargin=2em,xrightmargin=2em, aboveskip=1em, %设置边距
        tabsize=4, %设置tab空格数
        showspaces=false, %不显示空格
        basicstyle=\ttfamily,
       }
\begin{document}

%%%%% =======================================================================
\title{R语言-5}
\author{郑泽靖 \and zzjstat2023@163.com }
\institute{\normalsize 北京师范大学统计学院}
\date{\today}

% ===== title page =====
\begin{frame}[plain]
  \titlepage
\end{frame}

% ===== contents =====
% \begin{frame}
%  \frametitle{Outline}
%  \tableofcontents[hideallsubsections] %[pausesections]
% \end{frame}
\begin{frame}[fragile]{蒙特卡罗方法简介}
    \begin{block}{定义}
    蒙特卡罗 (Monte Carlo) 方法是一种以概率统计理论为指导的数值计算方法。它使用随机数(或伪随机数)来解决计算问题。
    \end{block}
    
    \begin{itemize}
        \item \textbf{核心思想}:当所求解的问题是某随机事件出现的概率,或者是某随机变量的数学期望时,通过“实验”的方法得出频率,以此逼近概率或期望。
        \item \textbf{应用场景}:数值积分,物理模拟等。
    \end{itemize}
\end{frame}

\begin{frame}[fragile]{经典案例:计算圆周率 $\pi$}
    \begin{columns}
        \column{0.6\textwidth}
            考虑单位正方形 $S$ 和其内切圆 $C$(第一象限):
            \[
            \begin{cases}
            0 \le X \le 1 \\
            0 \le Y \le 1
            \end{cases}
            \]
            若 $X, Y \sim U(0, 1)$ 独立同分布,则点落入圆内的概率为:
            \[
            P(X^2 + Y^2 \leq 1) = \frac{\text{圆面积}/4}{\text{正方形面积}} = \frac{\pi/4}{1} = \frac{\pi}{4}
            \]
        \column{0.4\textwidth}
            \begin{figure}
                \centering
                % 建议这里确保图片清晰,或使用TikZ绘制
                \includegraphics[width=\textwidth]{figs/1.png} 
            \end{figure}
    \end{columns}
\end{frame}

\begin{frame}[fragile]{R 代码实现:计算 $\pi$}
    基于大数定律,我们可以估计:
    \[
    \pi \approx 4 \times \frac{\#\{X_i^2 + Y_i^2 \leq 1\}}{n}
    \]
    
\begin{lstlisting}[language=R]
p_est <- function(n) {
    x <- runif(n); y <- runif(n)
    k <- (x^2 + y^2 <= 1)
    return(4 * sum(k) / n)
}

num <- c(1e3, 1e4, 1e5, 1e6) # 样本量
results <- c()
for (N in num) {
    results <- c(results, p_est(N))
}

data.frame(N=num, Est=results, 
           Error=abs(results - pi))
\end{lstlisting}
\end{frame} 

\begin{frame}[fragile]{理论基石:大数定律}
    \begin{block}{Kolmogorov 强大数定律}
    若 $X_1, X_2, \ldots, X_n$ 是独立同分布 (i.i.d.) 的随机变量且 $\mathrm{E}|X| < \infty$,则:
    \[
    P\left( \lim_{n \to \infty} \frac{1}{n} \sum_{k=1}^n X_k = \mathrm{E}(X) \right) = 1
    \]
    \end{block}
    \vspace{0.5cm}
    这意味着:只要样本量 $n$ 足够大,样本均值将几乎处处收敛于理论期望。
\end{frame}

\begin{frame}[fragile]{应用 1:估计期望 $\mathrm{E}(X)$}
    以 $X \sim U(0,1)$ 为例,理论值 $\mathrm{E}(X)=0.5$。
\begin{lstlisting}[language=R]
mean_unif <- function(n) {
    x <- runif(n, 0, 1)
    mean(x)
}
num <- c(10, 100, 1000, 10000, 100000)
results <- c()
for (N in num) {
    results <- c(results, mean_unif(N))
}
print(results)
\end{lstlisting}
    \small{随着 $n$ 增加,样本均值迅速逼近 0.5。}
\end{frame}

\begin{frame}[fragile]{应用 2:估计概率 $\mathrm{P}(A)$}
    概率可以看作是示性函数 $I_A(X)$ 的期望:
    \[
    \mathrm{P}(A) = \mathrm{E}[I_A(X)] \approx \frac{1}{n} \sum_{i=1}^n I_A(X_i) = \text{频率}
    \]
    以 $X \sim B(10, 0.5)$ 为例,估计 $P(X=4)$:
\begin{lstlisting}[language=R]
p_binom <- function(n) {
    x <- rbinom(n, size = 10, prob = 0.5)
    mean(x == 4) # 使用 mean() 计算逻辑向量的比例
}
results <- c()
for (N in num) {
    results <- c(results, p_binom(N))
}
\end{lstlisting}
    \small{理论值:`dbinom(4, 10, 0.5)` $\approx 0.205$。}
\end{frame}

\begin{frame}[fragile]{应用 3:估计复杂函数的期望}
    假设 $X_1, X_2 \sim N(0,1)$,估计 $\theta = E|X_1 - X_2|$。
    
    此处解析解较复杂,但蒙特卡罗模拟非常简单:
\begin{lstlisting}[language=R]
m <- 100000
x1 <- rnorm(m); x2 <- rnorm(m) # 向量化操作效率更高
g_val <- abs(x1 - x2)
est <- mean(g_val)

cat("估计值:", est, "\n")
cat("理论值:", 2/sqrt(pi)) 
\end{lstlisting}
    \small{理论值 $\approx 1.128379$。这种方法在处理多维复杂分布时优势巨大。}
\end{frame}

\begin{frame}{蒙特卡罗估计方法小结}
    我们可以通过生成随机样本来估计以下各类统计量:
    \begin{itemize}
        \item \textbf{期望值} $\mathrm{E}[X]$:直接计算样本均值 $\bar{x}$。
        \item \textbf{概率值} $\mathrm{P}(A)$:计算事件发生的频率(示性函数的均值)。
        \item \textbf{分布函数} $F(x)$:计算 $X_i \le x$ 的经验比例(Glivenko-Cantelli 定理保证了收敛性)。
        \item \textbf{积分} $\int g(x)dx$:转化为期望形式进行估计。
    \end{itemize}
\end{frame}
\begin{frame}[fragile]{蒙特卡罗积分:原理}
    目标:计算定积分 $\theta = \int_a^b g(x) dx$。
    
    \textbf{方法:} 引入在 $(a, b)$ 上的均匀分布随机变量 $X \sim U(a, b)$,其概率密度函数为 $f(x) = \frac{1}{b-a}$。
    
    我们可以将积分改写为期望的形式:
    \[
    \theta = \int_a^b g(x) dx = (b-a) \int_a^b g(x) \underbrace{\frac{1}{b-a}}_{f(x)} dx = (b-a) \mathrm{E}[g(X)]
    \]
    
    \textbf{估计量:}
    \[
    \hat{\theta} = \frac{b-a}{n} \sum_{i=1}^n g(X_i), \quad X_i \sim U(a, b)
    \]
\end{frame}

\begin{frame}[fragile]{积分示例 1:区间 $[0, 1]$}
    计算 $\theta = \int_0^1 e^{-x} dx$。此时 $b-a=1$。
\begin{lstlisting}[language=R]
m <- 10000
x <- runif(m)        # 默认 min=0, max=1
theta_hat <- mean(exp(-x))

print(paste("MC估计:", round(theta_hat, 5)))
print(paste("真实值:", round(1 - exp(-1), 5)))
\end{lstlisting}
\end{frame}

\begin{frame}[fragile]{积分示例 2:一般区间 $[2, 4]$}
    计算 $\theta = \int_2^4 e^{-x} dx$。注意这里要乘以区间长度 $(4-2)=2$。
\begin{lstlisting}[language=R]
m <- 10000
x <- runif(m, min = 2, max = 4)
# 公式: (b-a) * mean(g(x))
theta_hat <- 2 * mean(exp(-x)) 

print(paste("MC估计:", round(theta_hat, 5)))
print(paste("真实值:", round(exp(-2) - exp(-4), 5)))
\end{lstlisting}
\end{frame}

\begin{frame}{误差分析:中心极限定理 (CLT)}
    蒙特卡罗估计的误差服从什么分布?
    \begin{block}{Lindeberg-Lévy CLT}
    若 $X_1, \ldots, X_n$ 独立同分布,期望 $\mu$,方差 $\sigma^2 < \infty$,则:
    \[
    \sqrt{n}(\bar{X}_n - \mu) \xrightarrow{d} N(0, \sigma^2)
    \]
    或者标准化形式:
    \[
    \frac{\bar{X}_n - \mu}{\sigma / \sqrt{n}} \xrightarrow{d} N(0, 1)
    \]
    \end{block}
    这告诉我们,蒙特卡罗估计的误差收敛速度为 $O(1/\sqrt{n})$。
\end{frame}

\begin{frame}[fragile]{CLT 验证:可视化}
    验证:当 $n$ 较大时,标准化样本均值近似标准正态分布。
    以 $X \sim U(0,1)$ 为例,$\mu=0.5, \sigma^2=1/12$。
\begin{lstlisting}[language=R]
verify_clt <- function(m, n) {
    # m: 模拟次数, n: 每次的样本量
    z_scores <- numeric(m)
    true_sd <- sqrt(1/12)
    for (i in 1:m) {
        x <- runif(n)
        # 构造标准化变量
        z_scores[i] <- (mean(x)-0.5)/(true_sd/sqrt(n))
    }
    return(z_scores)
}
data <- verify_clt(m=2000, n=100)
\end{lstlisting}
\end{frame}

\begin{frame}[fragile]{CLT 验证:绘图结果}
    接上页代码,绘制直方图与理论曲线对比:
\begin{lstlisting}[language=R]
# 绘制频率直方图
hist(data, prob = TRUE, breaks = 40, 
     main = "Standardized Mean vs N(0,1)", 
     col = "lightblue", border = "white")

# 添加标准正态分布密度曲线
curve(dnorm(x), add = TRUE, col = "red", lwd = 2)
\end{lstlisting}
    \begin{itemize}
        \item 蓝色直方图代表模拟数据的分布。
        \item 红色曲线代表标准正态分布 $N(0,1)$。
        \item 两者高度重合说明了 CLT 的有效性。
    \end{itemize}
\end{frame}
% ===== 练习题部分 =====

\section{课后练习}

% 练习题 1:蒙特卡罗积分
\begin{frame}[fragile]{练习题 1:定积分计算}
    利用蒙特卡罗方法计算如下定积分的近似值,并与真实值进行比较:
    \[
    I = \int_{1}^{5} \ln(x) \, dx
    \]
    
    \textbf{提示:}
    \begin{itemize}
        \item 积分区间为 $[1, 5]$,长度为 4。
        \item 真实值计算公式:$\int \ln(x) dx = x\ln x - x$。
        \item 真实值结果约为 $5\ln 5 - 5 - (1\ln 1 - 1) \approx 4.047$。
    \end{itemize}

\end{frame}

% 练习题 2:多维概率估计
\begin{frame}[fragile]{练习题 2:多维随机变量概率估计}
    设 $X, Y, Z$ 相互独立且均服从 $U(0, 1)$ 分布。
    请编写 R 代码估计以下事件发生的概率:
    \[
    P(X + Y + Z \le 1)
    \]

    \textbf{思考:}
    \begin{itemize}
        \item 理论上,这是三维单位立方体中被平面 $x+y+z=1$ 截出的四面体的体积。
        \item 理论值为 $\frac{1}{3!} = \frac{1}{6} \approx 0.16667$。
    \end{itemize}
\end{frame}

\begin{frame}{}
\centering \Huge
  \emph{Thanks!}
\end{frame}

\end{document}
