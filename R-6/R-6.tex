\documentclass[9pt,compress,t,noamsthm,notheorem,table,handout]{ctexbeamer}
\usetheme{Boadilla}
\useinnertheme{circles}
\useoutertheme{shadow}
\usecolortheme{seahorse}
\usefonttheme[onlymath]{serif}
\setbeamertemplate{navigation symbols}{}
% \setbeamercovered{transparent}
\usepackage{natbib}
\usepackage{amsthm}
\usepackage{makecell}
\usepackage{multirow}
\renewcommand{\qedsymbol}{$\blacksquare$}
\bibliographystyle{unsrtnat}
\usepackage{color}
\setbeamercolor{myfootline}{bg=white,fg=blue}
\definecolor{myfoot}{rgb}{0.5,0.2,0.5}
\definecolor{darkblue}{rgb}{0.1,0,0.85}
\setbeamertemplate{headline}
  { \leavevmode\begin{beamercolorbox}[wd=\paperwidth,ht=1.25ex,dp=1ex,left]{}
    \end{beamercolorbox}}
\setbeamertemplate{footline}% 自定义页脚
  { \leavevmode\mbox{%
    \begin{beamercolorbox}[wd=.75\paperwidth,ht=2.25ex,dp=1ex,left]{myfootline}%
        \rule{2em}{0pt}\color{myfoot}\ttfamily\scriptsize%
        %\insertshortauthor~(\insertshortinstitute)
    \end{beamercolorbox}%
    \begin{beamercolorbox}[wd=.25\paperwidth,ht=2.25ex,dp=1ex,right]{myfootline}%
       {\color{myfoot}\ttfamily\scriptsize\insertframenumber{}/%
        \inserttotalframenumber\hspace*{3ex}}
    \end{beamercolorbox}}
    \vskip0pt }

\setbeamercolor{frametitle}{fg=blue,bg=white}
\setbeamertemplate{frametitle}{%
  \leavevmode\linespread{1}\large\textbf{\insertframetitle}\par
  \color{structure.fg!30!bg}\rule[6pt]{\linewidth}{2pt}\par\vspace{-1.0em}}

% \setbeamertemplate{blocks}[default] % beamer块(含定理类环境)不要阴影
\setbeamertemplate{bibliography entry title}{}{}
\setbeamertemplate{bibliography entry location}{}{}
\setbeamertemplate*{bibliography entry note}{}{}
\setbeamersize{text margin left=0.75cm, text margin right=0.75cm}

\setbeamercolor{bluebox}{fg=black,bg=blue!10}
\setbeamercolor{redbox}{fg=black,bg=red!10}
\newenvironment{Boxblue}[1][\textwidth]
  {\begin{beamercolorbox}[sep=0.1em,shadow=true,wd=#1,rounded=true,center]{bluebox}}
  {\end{beamercolorbox}}
\newenvironment{Boxred}[1][\textwidth]
  {\begin{beamercolorbox}[sep=0.1em,shadow=true,wd=#1,rounded=true,center]{redbox}}
  {\end{beamercolorbox}}
\usepackage{stmaryrd}
\usepackage{amsmath,amssymb,amsfonts,bm}
\usepackage{graphicx,xcolor}
\graphicspath{{figures/}}
\usepackage{hyperref}
\hypersetup{pdfborder=001,colorlinks=true,linkcolor=darkblue,urlcolor=blue}
\usepackage{bbding}
\newcommand{\Bullet}{{\fontsize{6pt}{6pt}\selectfont\CircleSolid}}
\newcommand{\Hand}{{\fontsize{8pt}{6pt}\selectfont\HandRight}}
\newcommand{\zhu}{{\color{blue!40}\Bullet}}
\newcommand{\zhuu}{{\color{red!80}\Hand}}
\newcommand{\labeli}{\zhu}
\newenvironment{blist}%
    {\begin{list}{{\hfill\raisebox{1.12pt}{\color{blue!60}\zhu}}}{%
     \leftmargin2em\labelwidth1.5em\labelsep0.5em
     \itemsep1ex\itemindent0pt\parsep0pt\topsep0pt}}
    {\end{list}}
\newenvironment{myitem}
  {\begin{list}{{\hfill\raisebox{0pt}{\labeli}}}{%
    \setlength{\leftmargin}{1.2em}\labelwidth0.8em\labelsep.4em%
    \itemsep1ex\parsep2pt\itemindent0pt\topsep0pt}}{\end{list}}
\newenvironment{subitem}
  {\begin{list}{{\hfill\raisebox{0pt}{-}}}{%
    \setlength{\leftmargin}{1.2em}\labelwidth0.8em\labelsep.4em%
    \itemsep0ex\parsep2pt\itemindent0pt\topsep0pt}}{\end{list}}
\usepackage{colortbl}
\usepackage{tikz}
\usetikzlibrary{arrows}
\usepackage{stmaryrd}
\usepackage{amsfonts}
\usepackage[ruled,linesnumbered]{algorithm2e}
\usepackage{float} 
\usepackage{booktabs}
\usepackage[framemethod=tikz]{mdframed}
\newmdenv[linecolor=green,middlelinewidth=1pt,%
          roundcorner=3pt,backgroundcolor=white,%
          innertopmargin=0.8em,innerbottommargin=0.5em,%
          innerleftmargin=3pt,innerrightmargin=3pt,%
          skipbelow=0.5em,skipabove=1em,%
          splittopskip=\topskip]{Block}
\newmdenv[linecolor=green,middlelinewidth=1pt,%
          roundcorner=3pt,backgroundcolor=red!5!white,%
          innertopmargin=0.5em,innerbottommargin=0.5em,%
          innerleftmargin=3pt,innerrightmargin=3pt,%
          skipbelow=0.5em,skipabove=1em,%
          splittopskip=\topskip]{redbox}
\newmdenv[linecolor=green,middlelinewidth=0.5pt,%
          %outerlinewidth=0.5pt,skipabove=0pt,
          roundcorner=3pt,backgroundcolor=white,%
          innerbottommargin=3pt,innerrightmargin=5pt,%
          innerleftmargin=5pt,leftmargin=0ex]{mathbox}
\newmdenv[linecolor=blue!5!green,middlelinewidth=0.5pt,%
          roundcorner=3pt,backgroundcolor=yellow!5,%
          % frametitle={Hello},frametitlebackgroundcolor=green!50,%
          % skipabove=2pt,skipbelow=2pt,%
          innerleftmargin=3pt,leftmargin=0ex]{notebox}
\newmdenv[linecolor=white,font={\scriptsize},%
          fontcolor=blue!85,backgroundcolor=yellow!5,%
          skipabove=1ex,skipbelow=0pt,innerbottommargin=0.5ex,%
          innerleftmargin=3pt,leftmargin=1em]{myref}

%%%%%%%%%%%%%%%%%%%%%%%%%%%%%%%%%%%%%%%%%%%%%%%%%%%%%%%%%%%%%%%%%%%%%%%%%%%%%%
\renewcommand{\thefootnote}{}% 不要编号
\setbeamertemplate{footnote}{% 首行不缩进
  \noindent\insertfootnotemark%
  \scriptsize\color{blue!85!green!85}\insertfootnotetext\par\kern1ex}
\renewcommand\footnoterule%    更改横线属性:长度,粗细,颜色
  {\color{red}\kern-3pt\rule{0.4\linewidth}{0.5pt}\par\kern2.6pt}
%%%%%%%%%%%%%%%%%%%%%%%%%%%%%%%%%%%%%%%%%%%%%%%%%%%%%%%%%%%%%%%%%%%%%%%%%%%%%%

\usepackage[many]{tcolorbox}
\tcbset{highlight math %
  style={enhanced, colframe=blue!40,colback=yellow!20,arc=4pt,boxrule=1pt}}
\newtcbox{\subsubtit}[1][]{%
  after skip=1em,boxrule=0.5pt,
  fontupper=\color{blue}\bfseries,top=0.5ex,bottom=0.5ex,
  left=1ex,right=1ex,
  colframe=green,colback=red!5!white,#1}

%\renewcommand{\baselinestretch}{1.1}
\linespread{1.1}
\setlength{\parskip}{1ex}

\usepackage{listings} %插入代码
\usepackage{xcolor}
\lstset{numbers=left, %设置行号位置
        numberstyle=\tiny, %设置行号大小
        keywordstyle=\color{blue}, %设置关键字颜色
        commentstyle=\color[cmyk]{1,0,1,0}, %设置注释颜色
        frame=single, %设置边框格式
        escapeinside=``, %逃逸字符(1左面的键),用于显示中文
        %breaklines, %自动折行
        extendedchars=false, %解决代码跨页时,章节标题,页眉等汉字不显示的问题
        xleftmargin=2em,xrightmargin=2em, aboveskip=1em, %设置边距
        tabsize=4, %设置tab空格数
        showspaces=false, %不显示空格
        basicstyle=\ttfamily,
       }

\title{R语言-6}
\author{郑泽靖 \and zzjstat2023@163.com}
\institute{北京师范大学统计学院}


\begin{document}
\begin{frame}
  \titlepage
\end{frame}
\section{数据读取与基础绘图}

% --- Frame 1: 读取 Excel ---
\begin{frame}[fragile]{从 Excel 文件中读取数据}
    在 R 中,推荐使用 \texttt{readxl} 包来读取 Excel 文件。
    
    \textbf{1. 安装和加载 \texttt{readxl} 包}:
\begin{lstlisting}[language=R]
install.packages("readxl")  # 安装 readxl 包
library(readxl)             # 加载 readxl 包
\end{lstlisting}

    \textbf{2. 读取 Excel 文件}:
\begin{lstlisting}[language=R]
# 设置文件路径 (注意使用正斜杠 / 或双反斜杠 \\)
file_path <- "path/to/your/file.xlsx"

# 读取 Excel 文件中的第一个工作表
data <- read_excel(file_path)

# 查看数据前几行
head(data)
\end{lstlisting}
\end{frame}

\section{连续变量的分布:直方图}

% --- Frame 2: 直方图概念 ---
\begin{frame}[fragile]{直方图 (Histogram)}
    直方图用于展示\textbf{连续变量}的分布情况。使用 \texttt{hist()} 函数即可绘制。
    
    例如,模拟生成 30 个服从正态分布的数据并绘制直方图:
\begin{lstlisting}[language=R]
x <- rnorm(30, mean=100, sd=1)
hist(x)
\end{lstlisting}

    \begin{figure}
        \centering
        % 建议使用 height 控制图片高度,避免溢出
        \includegraphics[height=0.45\textheight]{1.png} 
        \caption{模拟正态分布数据的直方图}
    \end{figure}
\end{frame}

% --- Frame 3: hist 参数详解 ---
\begin{frame}[fragile]{hist 函数详解}
    常用命令格式:
\begin{lstlisting}[language=R]
hist(x, breaks="Sturges", freq=FALSE)
\end{lstlisting}

    \begin{table}
    \centering
    \begin{tabular}{|l|l|}
    \hline 
    \textbf{参数} & \textbf{含义} \\
    \hline 
    \texttt{x} & 数据向量(必须为数值型) \\
    \hline 
    \texttt{breaks} & 设置分组边界 \\
    & (1) 默认为 "Sturges"(自动计算最适组距) \\
    & (2) 指定向量:自定义组边界点 \\
    \hline 
    \texttt{freq} & 纵轴显示设置 \\
    & \texttt{TRUE} (默认):显示\textbf{频数} (Frequency) \\
    & \texttt{FALSE}:显示\textbf{频率/密度} (Density) \\
    \hline
    \end{tabular}
    \end{table}
\end{frame}

% --- Frame 4: 自定义直方图 ---
\begin{frame}[fragile]{自定义直方图样式}
    可以使用 \texttt{main}、\texttt{xlab}、\texttt{ylab} 设置标签,用 \texttt{col} 设置颜色。
\begin{lstlisting}[language=R]
hist(x, 
     col = rainbow(15),     # 设置彩虹色
     main = '正态随机数分布', # 主标题
     xlab = '数值',           # x轴标签
     ylab = '频数')           # y轴标签
\end{lstlisting}

    \begin{figure}
        \centering
        \includegraphics[height=0.45\textheight]{3.png}
        \caption{自定义样式的直方图}
    \end{figure}
\end{frame}

% --- Frame 5: 频率直方图示例 ---
\begin{frame}[fragile]{直方图示例:频率直方图}
    \textbf{例一:绘制频率直方图 (Density)}
\begin{lstlisting}[language=R]
set.seed(123)
x <- rnorm(100, 80, 5)        # 生成100个均值为80的成绩数据
x <- x[x > 50 & x <= 100]     # 截取50-100分之间的数据
g <- seq(50, 100, 3)          # 定义组边界,组距为3

hist(x, breaks = g,           # 设置自定义组距
     freq = FALSE,            # FALSE 表示绘制频率(密度)直方图
     main = "学生成绩分布",    
     xlim = c(50, 100),       # 设置 x 轴范围
     xlab = "成绩",
     ylab = "频率/组距") 
\end{lstlisting}
\end{frame}

% --- Frame 6: 频率直方图图示 ---
\begin{frame}[fragile]{直方图示例:频率直方图}
    \begin{figure}
        \centering
        \includegraphics[height=0.7\textheight]{4.png}
        \caption{学生成绩的频率直方图}
    \end{figure}
\end{frame}

% --- Frame 7: 频数直方图示例 ---
\begin{frame}[fragile]{直方图示例:频数直方图}
    \textbf{例二:绘制频数直方图 (Frequency)}
\begin{lstlisting}[language=R]
set.seed(123)
x <- rnorm(100, 80, 5) 
x <- x[x > 50 & x <= 100]
g <- seq(50, 100, 3)

hist(x, breaks = g, 
     freq = TRUE,             # TRUE 表示绘制频数直方图
     main = "学生成绩分布", 
     xlim = c(50, 100), 
     xlab = "成绩",
     ylab = "频数") 
\end{lstlisting}
\end{frame}

% --- Frame 8: 频数直方图图示 ---
\begin{frame}[fragile]{直方图示例:频数直方图}
    \begin{figure}
        \centering
        \includegraphics[height=0.7\textheight]{2.png}
        \caption{学生成绩的频数直方图}
    \end{figure}
\end{frame}

\section{分类变量的分布:条形图与饼图}

% --- Frame 9: 频数统计 ---
\begin{frame}[fragile]{分类变量的频数统计}
    对于分类变量(因子),可以使用 \texttt{table()} 函数统计各类别的频数。
\begin{lstlisting}[language=R]
# 创建示例数据
x <- c("女", "男", "女", "女", "女", "男", "女", "男", 
       "女", "女", "男", "男", "女", "男", "女", "女", 
       "女", "女", "男", "男", "男", "男", "男")
y <- as.factor(x) 

u <- table(y)     # 计算频数
print(u)
# y
# 男  女 
# 12  17 
\end{lstlisting}

    使用 \texttt{addmargins()} 添加合计行:
\begin{lstlisting}[language=R]
addmargins(u)
# y   女  男  Sum
#    17  12  29
\end{lstlisting}
\end{frame}

% --- Frame 10: 频率统计 ---
\begin{frame}[fragile]{分类变量的频率统计}
    使用 \texttt{prop.table()} 计算比例(频率):
\begin{lstlisting}[language=R]
prop.table(u)
# 男        女 
# 0.4137931 0.5862069 
\end{lstlisting}

    结合 \texttt{addmargins()} 显示合计频率:
\begin{lstlisting}[language=R]
v <- prop.table(u)         
addmargins(v)              
# 男        女        Sum 
# 0.4137931 0.5862069 1.0000000
\end{lstlisting}
\end{frame}

% --- Frame 11: 条形图绘制 ---
\begin{frame}[fragile]{分类数据的条形图 (Bar Plot)}
    条形图和饼图常用于展示分类数据。在 R 中使用 \texttt{barplot()} 绘制条形图。

\begin{lstlisting}[language=R]
# 绘制频数条形图
barplot(u, main = "性别频数分布", ylab = "人数")  
\end{lstlisting}

    \begin{figure}
        \centering
        \includegraphics[height=0.6\textheight]{8.png}
        \caption{频数条形图}
    \end{figure}
\end{frame}

% --- Frame 13: 条形图展示2 ---
\begin{frame}[fragile]{分类数据的条形图}
    \begin{lstlisting}[language=R]
# 绘制频率条形图
barplot(v, main = "性别频率分布", ylab = "占比")  
\end{lstlisting}
    \begin{figure}
        \centering
        \includegraphics[height=0.7\textheight]{10.png}
        \caption{频率条形图}
    \end{figure}
\end{frame}

% --- Frame 14: 饼图 ---
\begin{frame}[fragile]{饼图 (Pie Chart)}
    饼图通过扇形面积比例展示各分类变量的频率。使用 \texttt{pie()} 函数绘制。
\begin{lstlisting}[language=R]
print(u)
# y
# 男 女 
# 12 17 

pie(u, main = "性别分布饼图")
\end{lstlisting}
    \begin{figure}
        \centering
        \includegraphics[height=0.5\textheight]{9.png}
    \end{figure}
\end{frame}

\section{多维分类变量的可视化}

% --- Frame 15: 案例引入 ---
\begin{frame}[fragile]{案例 4.5: 学生性别与专业分类}
    \textbf{背景}:某大学入学一年后分专业。调查了 69 名学生的志愿选择情况。
    
    \textbf{数据统计}:
    \begin{itemize}
        \item \textbf{文科}:29 人(男生 12,女生 17)
        \item \textbf{理科}:40 人(男生 19,女生 21)
    \end{itemize}
    
    此处涉及两个分类变量:
    \begin{itemize}
        \item 变量 $X$:专业(文科、理科)
        \item 变量 $Y$:性别(男、女)
    \end{itemize}
\end{frame}

% --- Frame 16: 条形图类型 ---
\begin{frame}[fragile]{多变量条形图}
    展示两个分类变量的关系时,常用的条形图类型:
    \begin{enumerate}
        \item \textbf{堆积条形图 (Stacked Bar Plot)}:分为等高(展示比例)和非等高(展示总量)。
        \item \textbf{并列条形图 (Dodged Bar Plot)}:便于直接比较各组数值。
    \end{enumerate}
    
    \begin{figure}
        \centering
        \includegraphics[width=0.8\textwidth]{denggao.png} 
        \caption{等高堆积条形图示例}
    \end{figure}
\end{frame}

% --- Frame 17: 数据准备 ---
\begin{frame}[fragile]{数据准备:创建列联表}
    首先将数据存储为矩阵或 table 对象:
\begin{lstlisting}[language=R]
# 创建矩阵数据
x <- matrix(c(12, 19, 17, 21), nrow=2, ncol=2)
colnames(x) <- c("男", "女")      # 列名:性别
rownames(x) <- c("文科", "理科")  # 行名:专业

y <- as.table(x)  # 转换为 table 对象
print(y)
\end{lstlisting}

    \begin{table}[h]
    \centering
    \begin{tabular}{lcc} 
    \hline
    & \textbf{男} & \textbf{女} \\
    \hline
    \textbf{文科} & 12 & 17 \\
    \textbf{理科} & 19 & 21 \\
    \hline
    \end{tabular}
    \caption{变量 y 的内容}
    \end{table}
\end{frame}

% --- Frame 18: 等高堆积条形图代码 ---
\begin{frame}[fragile]{绘制等高堆积条形图}
    等高堆积条形图主要用于比较\textbf{结构比例}(即条件概率)。
    
    \textbf{图 A:给定学科时的性别比例}
\begin{lstlisting}[language=R]
# prop.table(..., margin=2) 表示按列计算比例(列和为1)
barplot(prop.table(t(y), margin=2), 
        xlim = c(0, 3.5), 
        legend.text = colnames(y),     # 图例
        args.legend = list(x="topright"))
\end{lstlisting}

    \textbf{图 B:给定性别时的学科比例}
\begin{lstlisting}[language=R]
barplot(prop.table(y, margin=2), 
        xlim = c(0, 3.5), 
        legend.text = rownames(y), 
        args.legend = list(x="topright")) 
\end{lstlisting}
\end{frame}

% --- Frame 19: 条件概率解释 ---
\begin{frame}[fragile]{条件概率计算详解}
    函数 \texttt{prop.table(data, margin)} 用于计算比例:
    \begin{itemize}
        \item \texttt{margin=1}:按行计算(行和为1)。
        \item \texttt{margin=2}:按列计算(列和为1)。
    \end{itemize}

    \textbf{示例结果}:
    \begin{columns}
        \begin{column}{0.48\textwidth}
        \small
        \texttt{prop.table(t(y), margin=2)}
        \begin{tabular}{rrr} 
        \hline
        & 文科 & 理科 \\
        \hline
        男 & 0.41 & 0.48 \\
        女 & 0.59 & 0.52 \\
        \hline
        \end{tabular}
        \end{column}
        
        \begin{column}{0.48\textwidth}
        \small
        \texttt{prop.table(y, margin=2)}
        \begin{tabular}{rrr} 
        \hline
        & 男 & 女 \\
        \hline
        文科 & 0.39 & 0.45 \\
        理科 & 0.61 & 0.55 \\
        \hline
        \end{tabular}
        \end{column}
    \end{columns}
    
    \vspace{0.5cm}
    注:通过 \texttt{help(prop.table)} 可查看更多用法。
\end{frame}

% --- Frame 20: 堆积条形图 ---
\begin{frame}[fragile]{普通堆积条形图 (Stacked)}
    普通堆积条形图既能展示各组内部的比例结构,也能展示样本总量的差异。
    
    \begin{figure}
        \centering
        % 修正了文件名拼写 (假设原图名有误)
        \includegraphics[width=0.8\textwidth]{feidenggap.png} 
        \caption{非等高堆积条形图}
    \end{figure}
\end{frame}

% --- Frame 21: 堆积条形图代码 ---
\begin{frame}[fragile]{绘制堆积条形图}
    直接使用原始频数数据 \texttt{y} 进行绘制,无需计算比例。

    \textbf{绘制图 (a):按专业堆积}
\begin{lstlisting}[language=R]
barplot(t(y), 
        xlim = c(0, 3.5), 
        legend.text = colnames(y), 
        args.legend = list(x="topright"))
\end{lstlisting}

    \textbf{绘制图 (b):按性别堆积}
\begin{lstlisting}[language=R]
barplot(y, 
        xlim = c(0, 3.5), 
        legend.text = rownames(y), 
        args.legend = list(x="topright"))
\end{lstlisting}
\end{frame}

% --- Frame 22: 并列条形图 ---
\begin{frame}[fragile]{并列条形图 (Dodged)}
    并列条形图将各类别的柱子并排显示,便于直接比较数值大小。
    \begin{figure}
        \centering
        \includegraphics[width=0.8\textwidth]{binglie.png}
        \caption{并列条形图示例}
    \end{figure}
\end{frame}

% --- Frame 23: 并列条形图代码 ---
\begin{frame}[fragile]{绘制并列条形图}
    关键参数:\texttt{beside = TRUE}。

    \textbf{代码示例}:
\begin{lstlisting}[language=R]
# 图 (a)
barplot(t(y), 
        beside = TRUE,   # 关键参数:并列显示
        legend.text = colnames(y), 
        args.legend = list(x="topleft"))

# 图 (b)
barplot(y, 
        beside = TRUE, 
        legend.text = rownames(y), 
        args.legend = list(x="topleft"))
\end{lstlisting}
\end{frame}

% --- Frame 24: 练习题 ---
\begin{frame}{课堂练习}
    \textbf{练习题:}
    某统计学院对学生性别与专业进行调查,数据如下:
    
    \begin{table}[h]
    \centering
    \begin{tabular}{lccc}
    \hline
    \textbf{专业} & \textbf{男生 (人)} & \textbf{女生 (人)} & \textbf{合计} \\
    \hline
    文科 & 15 & 25 & 40 \\
    理科 & 20 & 30 & 50 \\
    工科 & 25 & 15 & 40 \\
    \hline
    \end{tabular}
    \end{table}

    \textbf{任务:} 请使用 R 语言完成以下图形绘制:
    \begin{enumerate}
        \item \textbf{堆积条形图}:展示每个专业中男、女生的\textbf{数量}构成。
        \item \textbf{等高堆积条形图}:展示每个专业中男、女生的\textbf{比例}情况。
        \item \textbf{并列条形图}:分组对比不同专业男、女生的数量。
    \end{enumerate}
\end{frame}

\begin{frame}{}
\centering \Huge
  \emph{Thanks!}
\end{frame}

\end{document}
