\documentclass[14pt,compress,t,noamsthm,notheorem,table,handout]{ctexbeamer}
\usetheme{Boadilla}
\useinnertheme{circles}
\useoutertheme{shadow}
\usecolortheme{seahorse}
\usefonttheme[onlymath]{serif}
\setbeamertemplate{navigation symbols}{}
% \setbeamercovered{transparent}
\usepackage{natbib}
\usepackage{amsthm}
\usepackage{makecell}
\usepackage{multirow}
\renewcommand{\qedsymbol}{$\blacksquare$}
\bibliographystyle{unsrtnat}
\usepackage{color}
\setbeamercolor{myfootline}{bg=white,fg=blue}
\definecolor{myfoot}{rgb}{0.5,0.2,0.5}
\definecolor{darkblue}{rgb}{0.1,0,0.85}
\setbeamertemplate{headline}
  { \leavevmode\begin{beamercolorbox}[wd=\paperwidth,ht=1.25ex,dp=1ex,left]{}
    \end{beamercolorbox}}
\setbeamertemplate{footline}% 自定义页脚
  { \leavevmode\mbox{%
    \begin{beamercolorbox}[wd=.75\paperwidth,ht=2.25ex,dp=1ex,left]{myfootline}%
        \rule{2em}{0pt}\color{myfoot}\ttfamily\scriptsize%
        %\insertshortauthor~(\insertshortinstitute)
    \end{beamercolorbox}%
    \begin{beamercolorbox}[wd=.25\paperwidth,ht=2.25ex,dp=1ex,right]{myfootline}%
       {\color{myfoot}\ttfamily\scriptsize\insertframenumber{}/%
        \inserttotalframenumber\hspace*{3ex}}
    \end{beamercolorbox}}
    \vskip0pt }

\setbeamercolor{frametitle}{fg=blue,bg=white}
\setbeamertemplate{frametitle}{%
  \leavevmode\linespread{1}\large\textbf{\insertframetitle}\par
  \color{structure.fg!30!bg}\rule[6pt]{\linewidth}{2pt}\par\vspace{-1.0em}}

% \setbeamertemplate{blocks}[default] % beamer块(含定理类环境)不要阴影
\setbeamertemplate{bibliography entry title}{}{}
\setbeamertemplate{bibliography entry location}{}{}
\setbeamertemplate*{bibliography entry note}{}{}
\setbeamersize{text margin left=0.75cm, text margin right=0.75cm}

\setbeamercolor{bluebox}{fg=black,bg=blue!10}
\setbeamercolor{redbox}{fg=black,bg=red!10}
\newenvironment{Boxblue}[1][\textwidth]
  {\begin{beamercolorbox}[sep=0.1em,shadow=true,wd=#1,rounded=true,center]{bluebox}}
  {\end{beamercolorbox}}
\newenvironment{Boxred}[1][\textwidth]
  {\begin{beamercolorbox}[sep=0.1em,shadow=true,wd=#1,rounded=true,center]{redbox}}
  {\end{beamercolorbox}}
\usepackage{stmaryrd}
\usepackage{amsmath,amssymb,amsfonts,bm}
\usepackage{graphicx,xcolor}
\graphicspath{{figures/}}
\usepackage{hyperref}
\hypersetup{pdfborder=001,colorlinks=true,linkcolor=darkblue,urlcolor=blue}
\usepackage{bbding}
\newcommand{\Bullet}{{\fontsize{6pt}{6pt}\selectfont\CircleSolid}}
\newcommand{\Hand}{{\fontsize{8pt}{6pt}\selectfont\HandRight}}
\newcommand{\zhu}{{\color{blue!40}\Bullet}}
\newcommand{\zhuu}{{\color{red!80}\Hand}}
\newcommand{\labeli}{\zhu}
\newenvironment{blist}%
    {\begin{list}{{\hfill\raisebox{1.12pt}{\color{blue!60}\zhu}}}{%
     \leftmargin2em\labelwidth1.5em\labelsep0.5em
     \itemsep1ex\itemindent0pt\parsep0pt\topsep0pt}}
    {\end{list}}
\newenvironment{myitem}
  {\begin{list}{{\hfill\raisebox{0pt}{\labeli}}}{%
    \setlength{\leftmargin}{1.2em}\labelwidth0.8em\labelsep.4em%
    \itemsep1ex\parsep2pt\itemindent0pt\topsep0pt}}{\end{list}}
\newenvironment{subitem}
  {\begin{list}{{\hfill\raisebox{0pt}{-}}}{%
    \setlength{\leftmargin}{1.2em}\labelwidth0.8em\labelsep.4em%
    \itemsep0ex\parsep2pt\itemindent0pt\topsep0pt}}{\end{list}}
\usepackage{colortbl}
\usepackage{tikz}
\usetikzlibrary{arrows}
\usepackage{stmaryrd}
\usepackage{amsfonts}
\usepackage[ruled,linesnumbered]{algorithm2e}
\usepackage{float} 
\usepackage{booktabs}
\usepackage[framemethod=tikz]{mdframed}
\newmdenv[linecolor=green,middlelinewidth=1pt,%
          roundcorner=3pt,backgroundcolor=white,%
          innertopmargin=0.8em,innerbottommargin=0.5em,%
          innerleftmargin=3pt,innerrightmargin=3pt,%
          skipbelow=0.5em,skipabove=1em,%
          splittopskip=\topskip]{Block}
\newmdenv[linecolor=green,middlelinewidth=1pt,%
          roundcorner=3pt,backgroundcolor=red!5!white,%
          innertopmargin=0.5em,innerbottommargin=0.5em,%
          innerleftmargin=3pt,innerrightmargin=3pt,%
          skipbelow=0.5em,skipabove=1em,%
          splittopskip=\topskip]{redbox}
\newmdenv[linecolor=green,middlelinewidth=0.5pt,%
          %outerlinewidth=0.5pt,skipabove=0pt,
          roundcorner=3pt,backgroundcolor=white,%
          innerbottommargin=3pt,innerrightmargin=5pt,%
          innerleftmargin=5pt,leftmargin=0ex]{mathbox}
\newmdenv[linecolor=blue!5!green,middlelinewidth=0.5pt,%
          roundcorner=3pt,backgroundcolor=yellow!5,%
          % frametitle={Hello},frametitlebackgroundcolor=green!50,%
          % skipabove=2pt,skipbelow=2pt,%
          innerleftmargin=3pt,leftmargin=0ex]{notebox}
\newmdenv[linecolor=white,font={\scriptsize},%
          fontcolor=blue!85,backgroundcolor=yellow!5,%
          skipabove=1ex,skipbelow=0pt,innerbottommargin=0.5ex,%
          innerleftmargin=3pt,leftmargin=1em]{myref}

%%%%%%%%%%%%%%%%%%%%%%%%%%%%%%%%%%%%%%%%%%%%%%%%%%%%%%%%%%%%%%%%%%%%%%%%%%%%%%
\renewcommand{\thefootnote}{}% 不要编号
\setbeamertemplate{footnote}{% 首行不缩进
  \noindent\insertfootnotemark%
  \scriptsize\color{blue!85!green!85}\insertfootnotetext\par\kern1ex}
\renewcommand\footnoterule%    更改横线属性:长度,粗细,颜色
  {\color{red}\kern-3pt\rule{0.4\linewidth}{0.5pt}\par\kern2.6pt}
%%%%%%%%%%%%%%%%%%%%%%%%%%%%%%%%%%%%%%%%%%%%%%%%%%%%%%%%%%%%%%%%%%%%%%%%%%%%%%

\usepackage[many]{tcolorbox}
\tcbset{highlight math %
  style={enhanced, colframe=blue!40,colback=yellow!20,arc=4pt,boxrule=1pt}}
\newtcbox{\subsubtit}[1][]{%
  after skip=1em,boxrule=0.5pt,
  fontupper=\color{blue}\bfseries,top=0.5ex,bottom=0.5ex,
  left=1ex,right=1ex,
  colframe=green,colback=red!5!white,#1}

%\renewcommand{\baselinestretch}{1.1}
\linespread{1.1}
\setlength{\parskip}{1ex}

\usepackage{listings} %插入代码
\usepackage{xcolor}
\lstset{numbers=left, %设置行号位置
        numberstyle=\tiny, %设置行号大小
        keywordstyle=\color{blue}, %设置关键字颜色
        commentstyle=\color[cmyk]{1,0,1,0}, %设置注释颜色
        frame=single, %设置边框格式
        escapeinside=``, %逃逸字符(1左面的键),用于显示中文
        %breaklines, %自动折行
        extendedchars=false, %解决代码跨页时,章节标题,页眉等汉字不显示的问题
        xleftmargin=2em,xrightmargin=2em, aboveskip=1em, %设置边距
        tabsize=4, %设置tab空格数
        showspaces=false, %不显示空格
        basicstyle=\ttfamily,
       }
\begin{document}

%%%%% =======================================================================
\title{R语言-3}
\author{郑泽靖 \and zzjstat2023@163.com }
\institute{\normalsize 北京师范大学统计学院}
\date{\today}

% ===== title page =====
\begin{frame}[plain]
  \titlepage
\end{frame}

% ===== contents =====
% \begin{frame}
%  \frametitle{Outline}
%  \tableofcontents[hideallsubsections] %[pausesections]
% \end{frame}
\begin{frame}{数据结构}
    \begin{figure}[htbp]
    \centering
    \includegraphics[width=0.8\linewidth]{figs/sjjg.PNG}
    \end{figure} 
\end{frame}
\section{列表(list)}
\begin{frame}[fragile]{列表的定义}
    列表可以看成向量结构的另一种推广。它允许其各个分量是任意的R语言结构,例如:
    \begin{itemize}
        \item 向量
        \item 矩阵
        \item 数据框
        \item 其他列表
    \end{itemize}

    列表结构能够将不同的对象以简单的形式组合在一起,方便编程者调用。
\end{frame}
\begin{frame}[fragile]{列表的特点}
    \begin{itemize}
        \item 列表中的每个元素可以是不同类型的R对象,具有更大的灵活性。
        \item 列表元素可以通过索引或名称进行访问。
        \item 许多R语言函数的计算结果都是以列表的方式表达的。
    \end{itemize}

\end{frame}
\begin{frame}[fragile]{生成列表}
    在R语言中,可以通过函数 \texttt{list()} 来生成列表数据。以下是示例代码:

    \begin{lstlisting}[language=R]
# 生成一个列表
x <- list(u = 2, v = "abcd")  # 列表有两个分量,分别为u和v
x # 显示列表x的内容
$u
[1] 2
$v
[1] "abcd"   
\end{lstlisting}
\end{frame}

\begin{frame}[fragile]{索引列表分量}    
    在R语言中,可以通过 \texttt{[[i]]} 或 \texttt{\$} 来索引和修改列表的分量。
    \begin{lstlisting}[language=R]
# 列表有两个分量:u 和 v
x <- list(u = 2, v = "abcd")  
# 索引x的第一分量
x$u    # 或者使用 x[[1]]
[1] 2
# 索引x的第二分量
x[[2]]
[1] "abcd"
\end{lstlisting}
\end{frame}

\begin{frame}[fragile]{修改列表分量}    
    \begin{lstlisting}[language=R]
# 将x的第一分量修改为1:3
x$u <- 1:3  # 修改为向量
# 显示x的第一分量的内容
x[[1]]
[1] 1 2 3
    \end{lstlisting}
\end{frame}

\begin{frame}[fragile]{names函数}
    在R语言中,函数 \texttt{names} 不仅能用于数据框,也能用于列表,查阅或修改各个分量的名称。
        \begin{lstlisting}[language=R]
# 创建列表
x <- list(u = 2, v = "abcd") 
names(x)
[1] "u"    "v"
names(x) <- c("num", "string") 
names(x) 
[1] "num"    "string"
    \end{lstlisting}
\end{frame}

\section{函数}
\begin{frame}[fragile]{函数}
    为了方便使用者,R语言将具有特定功能的程序代码封装在函数中。下面简单介绍一些常用函数的功能和使用方法。
\end{frame}

\begin{frame}[fragile]{1.求和函数}
函数 \texttt{sum} 计算向量的各个分量之和,矩阵的各行(列)元素之和使用函数 \texttt{rowSums} 和 \texttt{colSums}。
    \begin{lstlisting}[language=R]
# 计算向量c(1, 2, 3)的各分量之和
sum(c(1, 2, 3))  # 结果为 6
a <- matrix(1:6, 2, 3, byrow = TRUE)
# 计算 a 的各列之和,结果为3维行向量
b <- colSums(a)
# 计算 a 的各行之和,结果为2维行向量
c <- rowSums(a)
    \end{lstlisting}
\end{frame}


\begin{frame}[fragile]{2.函数的输入变量}
    通常,$R$ 语言的数学计算函数支持数据形式的输人变量(自变量),也支持向量形式的输入变量,还支持矩阵形式的输入变量.例如,运行程序代码
   \begin{lstlisting}[language=R]
x <- seq(1, 11, 2)
sin(x)
[1]  0.8414710  0.1411200 -0.9589243
[4]  0.65698660 0.4121185 -0.9999902
 \end{lstlisting}
\end{frame}
\begin{frame}[fragile]{3.绘图功能}
    R语言提供了方便的制图功能,可以绘制任何函数的图像。函数 \texttt{plot} 提供了一种绘制平面图形的功能,可以通过问号 \texttt{?} 来获取该函数的在线帮助。
    \begin{lstlisting}[language=R]
# 创建从0到10的数值序列,步长为0.1
x <- seq(0, 10, 0.1)
# 计算x的正弦值
y <- sin(x)
# 绘制函数曲线
plot(x,y,type="l",main="y=sin(x)",
xlab = "x", ylab = "y")
    \end{lstlisting}
\end{frame}
\begin{frame}[fragile]{关系运算符}
    在R语言中,关系运算符的计算结果是逻辑数据,表明参与运算的两个变量是否满足特定的关系。
    不仅两个数之间可以进行关系运算,向量和数之间、矩阵和数之间也可以进行关系运算。
    \begin{lstlisting}[language=R]
# 将向量 1:5 赋值给 X
X <- 1:5 
# 判断 X 的各个分量是否小于或等于 4
X <= 4  
[1] TRUE TRUE TRUE TRUE FALSE
    \end{lstlisting}
\end{frame}
\begin{frame}[fragile]{关系运算符的应用}
    可以利用关系运算符显示向量中满足特定条件的分量,或将这些分量统一改为特定的值。
    \begin{lstlisting}[language=R]
X <- 1:5 
# 显示 X 中小于 3 的分量构成的子向量
X[X < 3]  
[1] 1 2
# 将 X 中小于 3 的分量都改为 8
X[X < 3] <- 8 
X   # 再次显示 X 的结果
[1] 8 8 3 4 5
    \end{lstlisting}
\end{frame}
\begin{frame}[fragile]{其他关系运算示例}
    \begin{lstlisting}[language=R]
# 将字符 "abc" 赋值给 b
b <- "abc"
# 将 b 赋值给 c
c <- b 
# 判断b中的内容是否和字符"abc"相等
b == "abc"  
[1] TRUE
# 判断b存储的内容是否和c的相等
b == c  
[1] TRUE
    \end{lstlisting}
\end{frame}

\begin{frame}{离散均匀分布}
    若随机变量 \(X\) 满足:
    \[
    \mathrm{P}(X=k) = \frac{1}{n}, \quad k=1,2,\ldots,n
    \]
    则称 \(X\) 服从以 \(n\) 为参数的离散均匀分布。

    以 \(n\) 为参数的离散均匀分布随机变量的观测值称为以 \(n\) 为参数的离散均匀分布随机数。
\end{frame}

\begin{frame}[fragile]{离散均匀分布在R语言中的模拟}
    可以使用函数 \texttt{sample} 来模拟离散均匀分布随机数。其简单调用方式为:
    \begin{lstlisting}[language=R]
sample(x, # 等可能地选择 x 的分量
size,  # size为要选取的变量的个数
replace = FALSE)  
# replace为真,有放回抽样;
#否则,无放回抽样
    \end{lstlisting}
    \end{frame}




    \begin{frame}[fragile]{逻辑运算符}
逻辑运算符用于逻辑变量的运算。常用的逻辑运算符及其含义如下:
\begin{itemize}
    \item \texttt{!}:非运算
    \item \texttt{\&}:与运算
    \item \texttt{|}:或运算
\end{itemize}
\end{frame}

\begin{frame}[fragile]{逻辑运算符示例}
\begin{lstlisting}[language=R]
!TRUE    # 非运算结果为 FALSE
!FALSE   # 非运算结果为 TRUE
TRUE & TRUE   # 与运算结果为 TRUE
TRUE & FALSE  # 与运算结果为 FALSE
FALSE & FALSE # 与运算结果为 FALSE
TRUE | TRUE   # 或运算结果为 TRUE
TRUE | FALSE  # 或运算结果为 TRUE
FALSE | FALSE # 或运算结果为 FALSE
\end{lstlisting}
\end{frame}

\begin{frame}[fragile]{逻辑矩阵的示例}
\begin{lstlisting}[language=R]
> x
     [,1] [,2] [,3]
[1,]    1    3    5
[2,]    2    4    6

> a <- x <= 3 
> a
       [,1]  [,2]  [,3]
[1,]  TRUE  TRUE FALSE
[2,]  TRUE FALSE FALSE
\end{lstlisting}
\end{frame}

\begin{frame}[fragile]{逻辑矩阵的示例}
\begin{lstlisting}[language=R]
> b <- x > 2  
> b
       [,1]  [,2]  [,3]
[1,] FALSE FALSE  TRUE
[2,]  TRUE  TRUE  TRUE

> !a
       [,1]  [,2]  [,3]
[1,] FALSE FALSE  TRUE
[2,] FALSE  TRUE  TRUE
\end{lstlisting}
\end{frame}

\begin{frame}[fragile]{逻辑矩阵的示例}
\begin{lstlisting}[language=R]
> a & b
       [,1]  [,2]  [,3]
[1,] FALSE FALSE FALSE
[2,]  TRUE FALSE FALSE

> a | b
       [,1]  [,2]  [,3]
[1,]  TRUE  TRUE  TRUE
[2,]  TRUE  TRUE  TRUE
\end{lstlisting}
\end{frame}


\begin{frame}[fragile]{短路逻辑或}
\texttt{||} 是短路逻辑或(OR)运算符,也用于将两个逻辑表达式组合在一起。它与 \texttt{|} 的区别在于,只要第一个表达式为TRUE,就会返回TRUE,并且不会评估第二个表达式。
\begin{lstlisting}[language=R]
x <- 2
a <- x <= 3                 
b <- x > 2                 
result <- a && b           
\end{lstlisting}
\end{frame}

\begin{frame}[fragile]{短路逻辑和}
\texttt{\&\& }是短路逻辑与(AND)运算符,也用于将两个逻辑表达式组合在一起。它与 \texttt{\&} 的区别在于,只要第一个表达式为FALSE,就会返回FALSE,并且不会评估第二个表达式。
\begin{lstlisting}[language=R]
x <- 2
a <- x <= 3                 
b <- x > 2                  
result <- a || b            
\end{lstlisting}
\end{frame}


    \begin{frame}[fragile]{作业}
\begin{itemize}
    \item[1.]使用R语言创建一个数据框,包含以下列:
\begin{itemize}
    \item \texttt{ID}:1到10的整数
    \item \texttt{Name}:随机给定10个名字(如“学生1”,“学生2”)
    \item \texttt{Score}:随机生成10个1到100之间的整数
\end{itemize}
    \item[2.] 计算数据框中所有学生的平均分,并输出平均分。
    \item[3.] 从数据框中筛选出分数大于等于70的学生,并输出这些学生的信息。
    \item[4.] 从数据框中随机抽取5个学生的\texttt{Score},使用不放回的方式抽样,并打印结果。
\end{itemize}
\end{frame}

\begin{frame}{}
\centering \Huge
  \emph{Thanks!}
\end{frame}

\end{document}
