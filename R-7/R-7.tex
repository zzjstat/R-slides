\documentclass[9pt]{ctexbeamer}

%~~~~~~~~~~~~~~~~~~~~~~~~~~~~~~~~~~~~~~~~~~~~~~~

%~~~~~~~~~~~~~~~~~~~~~~~~~~~~~~~~~~~~~~~~~~~~~~~~~~~~~~~~~~~~~~~~~~~~~~~~~~~~~~

%~~~~~~~~~~~~~~~~~~~~~~~~~~~~~~~~~~~~~~~~~~~~~~~~~~~~~~~~~~~~~~~~~~~~~~~~~~~~~~
% Define where theme files are located. ('/styles')
\usepackage{styles/fluxmacros}
\usefolder{styles}
% Use Flux theme v0.1 beta
% Available style: asphalt, blue, red, green, gray 
\usetheme[style=asphalt]{flux}
%~~~~~~~~~~~~~~~~~~~~~~~~~~~~~~~~~~~~~~~~~~~~~~~~~~~~~~~~~~~~~~~~~~~~~~~~~~~~~~
\def\pgfsysdriver{pgfsys -dvipdfmx.def}
\usepackage{tikz}
%~~~~~~~~~~~~~~~~~~~~~~~~~~~~~~~~~~~~~~~~~~~~~~~~~~~~~~~~~~~~~~~~~~~~~~~~~~~~~~
% Extra packages for the demo:
\usepackage{booktabs}
\usepackage{amsmath}
\usepackage{colortbl}
\usepackage{ragged2e}
\usepackage{schemabloc}
\usepackage{graphicx}
%~~~~~~~~~~~~~~~~~~~~~~~~~~~~~~~~~~~~~~~~~~~~~~~~~~~~~~~~~~~~~~~~~~~~~~~~~~~~~~
% 导入包
\usepackage{hyperref}
% 格式设置
\hypersetup{hidelinks,
	colorlinks=true,
	allcolors=cyan,
	pdfstartview=Fit,
	breaklinks=true}

 %代码
\usepackage{listings} %插入代码
\usepackage{xcolor}
\lstset{numbers=left, %设置行号位置
        numberstyle=\tiny, %设置行号大小
        keywordstyle=\color{blue}, %设置关键字颜色
        commentstyle=\color[cmyk]{1,0,1,0}, %设置注释颜色
        frame=single, %设置边框格式
        escapeinside=``, %逃逸字符(1左面的键),用于显示中文
        %breaklines, %自动折行
        extendedchars=false, %解决代码跨页时,章节标题,页眉等汉字不显示的问题
        xleftmargin=2em,xrightmargin=2em, aboveskip=1em, %设置边距
        tabsize=4, %设置tab空格数
        showspaces=false, %不显示空格
        basicstyle=\ttfamily,
       }


%~~~~~~~~~~~~~~~~~~~~~~~~~~~~~~~~~~~~~~~~~~~~~~~~~~~~~~~~~~~~~~~~~~~~~~~~~~~~~~
% Informations
\title{R-7}
\subtitle{}
\author{郑泽靖 \and zzjstat2023@163.com}
\institute{北京师范大学统计学院}
\date{\today}
\titlegraphic{assets/stat.png}
\usepackage{wrapfig}
\usepackage{caption}
\begin{document}
\titlepage
\begin{frame}[fragile]{集中趋势的分析}
\begin{lstlisting}[language=R]
1、样本均值  mean()
2、样本中位数 median()
3、样本分位数 quantile(x,a,type=5)  
type在1:9取值,代表9种计算分位数的方法
4、众数
#连续型变量样本数据的众数
x=runif(100,5,10)
hist(x)
y <- hist(x,plot=F) #计算x的分组数据统计结果
maxID <- which.max(y$counts) #计算最大频数所在区间序号
mean(y$breaks[maxID:(maxID+1)])#计算第maxID区间的中心
#离散型变量样本数据的众数
x=sample(1:10,100,T)
y <- table(x) #计算x的分组数据统计结果
names(which.max(y)) #计算最大频数
\end{lstlisting}
\end{frame}
\begin{frame}[fragile]{集中趋势的分析}
\begin{lstlisting}[language=R]
5.方差: var(x)
x<-c (42,55,64,70,75,78,80,82,82,82, 
 85,85,85,85,88,90,90,92,95,99)
y<-c(39,52,61,68,72,76,77,78,79,78, 
 83,83,81,81,85,87,86,91,91,98)
var(x)
var(y)
6.标准差:sd (x) 
sd(x)
7.标准化 scale(x)
\end{lstlisting}
\end{frame}
\begin{frame}[fragile]{QQ图}
使用 \texttt{qqnorm} 和 \texttt{qqline} 绘制正态 QQ 图。当样本来自正态分布时,散点应近似分布在一条直线附近。

QQ 图的做法如下:设有 $n$ 个观测值 $y_1, y_2, \ldots, y_n$,从小到大排列。对于样本来自 $N(\mu, \sigma^2)$,有:
\[
y_i \approx \mu + \sigma x_i
\]
其中 $x_i$ 是标准正态分布的分位数。

R 代码示例:
\begin{lstlisting}[language=R]
qqnorm(x)
qqline(x, lwd=2, col='blue')
\end{lstlisting}
散点越集中在对角线附近,数据越可能服从正态分布。
\end{frame}
\begin{frame}{QQ图}
\begin{figure}
    \centering
    \includegraphics[scale=0.6]{12.png}
\end{figure}
\end{frame}
\begin{frame}[fragile]{QQ图}
下面的程序模拟对数正态数据,并作正态QQ图:
 \begin{lstlisting}[language=R]
z <- 10^rnorm(30, mean=0, sd=0.2)
qqnorm(z)
qqline(z, lwd=2, col='blue')
\end{lstlisting}
\begin{figure}
    \centering
    \includegraphics[scale=0.4]{13.png}
\end{figure}
\end{frame}

\begin{frame}{五数概括}
    \begin{figure}
    \centering
    \includegraphics[scale=0.7]{wushu.png}
\end{figure}
\end{frame}

\begin{frame}[fragile]{盒形图}
 \begin{lstlisting}[language=R]
x <-c(42,55,64,70,75,78,80,82,82,82,85,85,85,85,85,88
,90,90,92,95,87)
fivenum(x) #计算五数概括:四分位数,最小值,最大值
boxplot(x,range=1.5,horizontal=F,ylab="分数")
\end{lstlisting}
其中:
range:默认1.5,表示上、下两条须线的长度不超过四分位间距(上四分位数-下四分位数)的1.5倍
horizontal=T,则为水平的盒形图
x:也可以是数据框,对其中的每一个变量画盒形图
\end{frame}
\begin{frame}[fragile]{盒形图}
\begin{figure}
    \centering
    \includegraphics[scale=0.4]{14.png}
\end{figure}
\end{frame}
\begin{frame}[fragile]{盒形图}
使用 \texttt{boxplot} 函数可以在同一坐标系下绘制来自不同总体的样本数据的盒形图,方便比较它们的分布特征和离群数据情况。只需将不同的样本数据放入一个数据框,然后将该数据框作为 \texttt{boxplot} 的输入变量。

例如,当数据框 \texttt{xy} 的各列为不同总体的样本数据时,使用以下代码绘制盒形图:

\begin{lstlisting}[language=R]
boxplot(xy)
例:
xy <- data.frame("甲班"=x, "乙班"=y)
boxplot(xy, range=1.5, ylab="分数")
\end{lstlisting}
\end{frame}
\begin{frame}[fragile]{盒形图}
\begin{figure}
    \centering
    \includegraphics[scale=0.7]{hx 2.png}
\end{figure}
\end{frame}
\section{极大似然估计}
\begin{frame}[fragile]{apply函数}
 \begin{lstlisting}[language=R]
x <- matrix(rnorm(1000*5),1000,5)
#生成1000组容量为5的样本,
每个样本都服从标准正态分布,排列成1000行5列的矩阵。
apply(x,MARGIN=1,FUN=mean) 
#对x的每一行求均值
\end{lstlisting}
apply函数能够极大减少 for 循环的使用:
 \begin{lstlisting}[language=R]
apply(X, MARGIN, FUN, ...)
\end{lstlisting}
\begin{itemize}
    \item x:     运算对象:数组、矩阵、数据框,数据至少是二维的
    \item MARGIN:取1,2,c(1,2)分别表示对运算对象的行、列、同时对二者运算
    \item FUN:  指定运算函数,如mean表示求均值
\end{itemize}
\end{frame}
\begin{frame}[fragile]{apply函数}
 \begin{lstlisting}[language=R]
> (my.matrx <- matrix(c(1:10, 11:20, 21:30), 
nrow = 10, ncol = 3))
      [,1] [,2] [,3]
 [1,]    1   11   21
 [2,]    2   12   22
 [3,]    3   13   23
 [4,]    4   14   24
 [5,]    5   15   25
 [6,]    6   16   26
 [7,]    7   17   27
 [8,]    8   18   28
 [9,]    9   19   29
[10,]   10   20   30
\end{lstlisting}
\end{frame}
\begin{frame}[fragile]{apply函数}
 \begin{lstlisting}[language=R]
#对每一行求和
> apply(my.matrx, 1, sum)
 [1] 33 36 39 42 45 48 51 54 57 60
#计算每一列的长度
> apply(my.matrx, 2, length)
[1] 10 10 10
#传递自定义匿名函数
> apply(my.matrx, 2, function (x) length(x)-1)
[1] 9 9 9
#使用定义在外部的函数
> st.err <- function(x){
+     sd(x)/sqrt(length(x))
+ }
> apply(my.matrx,2, st.err)
[1] 0.9574271 0.9574271 0.9574271
\end{lstlisting}
\end{frame}
\begin{frame}[fragile]{apply函数}
 \begin{lstlisting}[language=R]
> (my.matrx2 <- apply(my.matrx, 1:2, function(x) x+3))
      [,1] [,2] [,3]
 [1,]    4   14   24
 [2,]    5   15   25
 [3,]    6   16   26
 [4,]    7   17   27
 [5,]    8   18   28
 [6,]    9   19   29
 [7,]   10   20   30
 [8,]   11   21   31
 [9,]   12   22   32
[10,]   13   23   33
\end{lstlisting}
\end{frame}


\begin{frame}[fragile]{最大似然估计}
R函数optim()、nlm()、optimize()可以用来求函数极值, 因此可以用来计算最大似然估计。 optimize()只能求一元函数极值。\\
正态分布最大似然估计有解析表达式。 作为示例, 用R函数进行数值优化求解。

对数似然函数为:
$$\ln L\left(\mu, \sigma^2\right)=-\frac{n}{2} \ln (2 \pi)-\frac{n}{2} \ln \sigma^2-\frac{1}{2 \sigma^2} \sum\left(X_i-\mu\right)^2$$

定义R的优化目标函数为上述对数似然函数去掉常数项以后乘以, 求其最小值点。
\end{frame}
\begin{frame}[fragile]{最大似然估计}
目标函数为:
 \begin{lstlisting}[language=R]
objf.norm1 <- function(theta, x){
  mu <- theta[1]
  s2 <- exp(theta[2])
  n <- length(x)
  res <- n*log(s2) + 1/s2*sum((x - mu)^2)
  res
}
\end{lstlisting}
其中 $\theta_1$ 为均值参数 $\mu , \theta_2$ 为方差参数 $\sigma^2$ 的对数值。 $\mathrm{x}$ 是样本数值组成的 $R$ 向量。
\end{frame}

\begin{frame}[fragile]{optim函数}
可以用optim函数来求极小值点。下面是一个模拟演示:
 \begin{lstlisting}[language=R]
  n<-30
  mu0 <- 20
  sigma0 <- 2
  set.seed(1)
  x <- rnorm(n, mu0, sigma0)
  theta0 <- c(0,0)
  ores <- optim(theta0, objf.norm1, x=x)
  print(ores)
  theta <- ores$par
  mu <- theta[1]
  sigma <- exp(0.5*theta[2])
\end{lstlisting}
\end{frame}
\begin{frame}[fragile]{optim函数}
 \begin{lstlisting}[language=R]
## $par
## [1] 20.166892  1.193593
## 
## $value
## [1] 65.83709
## 
## $counts
## function gradient 
##      115       NA 
## 
## $convergence
## [1] 0
## 
## $message
## NULL
## 
## 真实mu= 20  公式估计mu= 20.16492  
## 数值优化估计mu= 20.16689 
## 真实sigma= 2 公式估计sigma= 1.817177  
## 数值优化估计sigma= 1.816291
\end{lstlisting}
\end{frame}

\begin{frame}[fragile]{optimize()}
函数 optimize() 进行一元函数数值优化计算。例如,在一元正态分布最大似然估计中,在对数似然函数中代入 $\mu=\bar{x}$ ,目标函数 :
$$
h\left(\sigma^2\right)=\ln \left(\sigma^2\right)+\frac{1}{\sigma^2} \sum\left(X_i-\bar{X}\right)^2
$$
下面的例子模拟计算 $\sigma^2$ 的最大似然估计:
 \begin{lstlisting}[language=R]
  n<-30
  mu0 <- 20;  sigma0 <- 2;  set.seed(1)
  x <- rnorm(n, mu0, sigma0); mu <- mean(x)
  ss <- sum((x - mu)^2)/length(x)
  objf <- function(delta, ss) log(delta) + 1/delta*ss
  ores <- optimize(objf, lower=0.0001,
                   upper=1000, ss=ss)
  delta <- ores$minimum;sigma <- sqrt(delta)
  print(ores)
\end{lstlisting}
\end{frame}
\begin{frame}[fragile]{optimize()}
 \begin{lstlisting}[language=R]
## $minimum
## [1] 3.30214
## 
## $objective
## [1] 2.194568
## 
## 真实sigma= 2 公式估计sigma= 1.817177  
## 数值优化估计sigma= 1.817179
\end{lstlisting}
该函数的计算结果会存在列表中,是有两个分量的列表变量:
 \begin{lstlisting}[language=R]
#第一个分量$maximum:最优的参数取值
#第二个分量$objective:最优(最大或最小)的函数值
\end{lstlisting}
\end{frame}
\begin{frame}[fragile]{矩估计}
1、矩估计矩分为\\
(1) 原点矩 $\mu_k=E\left(X^k\right)$,均值: $\mu=E(X), k=1$\\
(2) 中心矩 $v_k=E(X-E(X))^k$, 方差: $v_2=E(X-E(X))^2$\\
2.矩估计即\\
(1) 用样本矩估计总体矩\\
(2) 用样本矩的函数估计总体矩的函数\\
$\rightarrow$ 把所有的E都变成求均值\\
例子: 用样本均值 $\bar{x}=\frac{1}{n} \sum_{i=1}^n x_i$ 估计总体均值 $u$
\end{frame}


\begin{frame}{练习}
1.生成100个来自 $x \sim N(2,4)$的随机变量值,计算其均值和标准差。重复试验100次,绘制均值和标准差的QQ图观察其结果 .\\
\end{frame}
\begin{frame}[fragile]{示例}
 \begin{lstlisting}[language=R]
# 进行100次实验
n_trials <- 100
means <- numeric(n_trials)  # 用于存储均值
sds <- numeric(n_trials)     # 用于存储标准差

# 重复试验100次
for (i in 1:n_trials) {
  sample_data <- rnorm(100, mean = 2, sd = 2)  # 生成100个来自N(2, 4)的随机变量
  means[i] <- mean(sample_data)  # 计算均值
  sds[i] <- sd(sample_data)       # 计算标准差
}

# 绘制均值的QQ图
par(mfrow=c(1, 2))  # 设置图形窗口为1行2列
qqnorm(means, main = "均值的QQ图")
qqline(means, col='blue', lwd=2)

# 绘制标准差的QQ图
qqnorm(sds, main = "标准差的QQ图")
qqline(sds, col='red', lwd=2)
\end{lstlisting}
\end{frame}
\begin{frame}{}
\centering \Huge
  \emph{Thanks!}
\end{frame}
\end{document}